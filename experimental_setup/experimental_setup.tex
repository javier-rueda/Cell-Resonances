\documentclass[a4paper, 12pt]{article}

\usepackage{graphicx}
\usepackage{geometry}
\usepackage{amsmath}
\usepackage{hyperref}

\geometry{margin=1in}

\title{Experimental Setup Documentation}
\date{ }

\begin{document}
	
	\maketitle
	
	\section*{Overview}
	
	This document provides detailed information about the \textbf{experimental setup} used in the \textit{"Cellular Vibration Analysis using Atomic Force Microscopy (AFM)"} project.
	
	\section*{AFM Model}
	
	The AFM used in this project is the NanoWizard 5 from Bruker BioAFM.
	
	\section*{Signal Acquisition}
	
	The Atomic Force Microscope (AFM) operates by scanning a sharp tip (attached to a cantilever) across a sample. As the tip interacts with the surface, forces between the tip and the sample cause the cantilever to deflect. The deflection is typically measured in two directions: "Lateral Deflection" and "Vertical Deflection.".\\
	
	The signals are transmitted into a Low-Noise Preamplifier "SR560" from Stanford Research Systems. There are two SR560 units, one for the lateral deflection and another for the vertical deflection. The SR560 serves both as a filter and amplifier, enhancing the quality of the signals.\\
	
	The SR560 voltage preamplifier provides the following functionalities.
	
	(https://www.thinksrs.com/products/sr560.htm):
	
	\begin{itemize}
		\item Input Noise: 4 nV/$\sqrt{\text{Hz}}$
		\item Bandwidth: 1 MHz
		\item Variable Gain: 1 to 50,000
		\item AC or DC Coupled
		\item Two Configurable Signal Filters
		\item Differential and Single-Ended Inputs
		\item Line or Battery Operation
		\item RS-232 Interface
	\end{itemize}
	
	\section*{Data Acquisition}
	
	The conditioned signals from the SR560 units are then transmitted into a high-density NI PXI-5105 oscilloscope for data acquisition.	The PXI-5105 allows for detailed analysis of the signals by using the LabView software installed in the computer, capturing the dynamic behavior of the cantilever during scanning. 
	
	
	
	\subsection*{Specifications of PXI-5105}
	
	\begin{itemize}
		\item 60 MHz bandwidth
		\item 8 channels
		\item 12-bit resolution
		\item Sample rate: Up to 60 MS/s
	\end{itemize}
	
	
	
	
	
	
\end{document}
